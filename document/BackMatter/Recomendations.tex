%===================================================================================
% Chapter: Recomendaciones
%===================================================================================
\chapter*{Recomendaciones}\label{chapter:recomendations}
\addcontentsline{toc}{chapter}{Recomendaciones}


A partir de los hallazgos y limitaciones identificadas, se 
proponen las siguientes recomendaciones para futuras 
investigaciones:
\begin{itemize}
    \item \textbf{Mejorar la sincronización y validación}: 
    Incluir pulsos acústicos artificiales (tonos de referencia) 
    al inicio y final de cada grabación para facilitar la 
    alineación temporal. Paralelamente, colaborar con 
    bioacústicos expertos en \textit{E. eileenae} para generar 
    un \textit{ground truth} robusto mediante etiquetado manual 
    estandarizado, permitiendo métricas de evaluación más 
    rigurosas.
    
    \item \textbf{Ampliar la diversidad de datos}: Replicar el 
    estudio en múltiples temporadas reproductivas y localidades 
    geográficas, incorporando variables ambientales 
    (temperatura, humedad) para analizar su influencia en la 
    estructura de los coros. Esto enriquecería la generalidad de 
    los modelos propuestos.

    \item \textbf{Integrar información espacial}: Utilizar 
    micrófonos direccionales o arreglos de sensores para 
    triangular posiciones exactas de los individuos, combinando 
    datos acústicos con coordenadas georreferenciadas. Esta 
    información mejoraría la asignación de cantos y permitiría 
    modelar interacciones en función de distancias físicas.

    \item \textbf{Extender el modelo de interacciones}: Explorar modelos 
    alternativos como redes neuronales recurrentes o sistemas de 
    ecuaciones diferenciales acopladas, capaces de capturar 
    dependencias temporales y no lineales. Incorporar términos 
    de orden superior en el modelo de Ising o evaluar 
    formulaciones no estacionarias (fuera del equilibrio termodinámico) 
    podría mejorar la 
    representación de la dinámica coral.

    \item \textbf{Analizar la regulación frecuencial como estrategia de segregación acústica}: 
    Profundizar en el estudio de las frecuencias características 
    de los cantos individuales de \textit{E. eileenae}, 
    evaluando si existen ajustes dinámicos que minimicen la 
    superposición espectral (multiplexación) dentro del coro.
    

\end{itemize}


Estas mejoras metodológicas, junto con la adopción de marcos 
teóricos más flexibles, potenciarían el estudio cuantitativo de 
sistemas bioacústicos complejos, tanto en anuros como en otras 
especies gregarias.