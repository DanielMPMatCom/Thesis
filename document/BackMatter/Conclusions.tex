%===================================================================================
% Chapter: Conclusiones
%===================================================================================
\chapter*{Conclusiones}\label{chapter:conclusions}
\addcontentsline{toc}{chapter}{Conclusiones}


El presente estudio desarrolló y validó un flujo computacional 
automatizado para el análisis de coros de \textit{Eleutherodactylus 
eileenae} a partir de grabaciones bioacústicas. Los objetivos 
específicos se abordaron mediante tres etapas principales: 
eliminación de ruido y sincronización de señales, detección y 
asignación de cantos, e 
inferencia de interacciones acústicas. La sincronización de las 
nueve pistas de audio se logró mediante 
correlación cruzada sobre los histogramas de tiempos calculados a partir de 
los puntos de máximos locales en los Mel-espectrogramas, 
propiciando la correcta aplicación de los métodos de detección.
Esto, sumado a la eliminación de ruido, dio como resultado un 
dataset limpio y sincronizado que se podrá continuar utilizando en el futuro.

Los algoritmos heurísticos propuestos para detectar y asignar 
cantos demostraron alta consistencia interna, con correlaciones 
superiores elevadas entre ejecuciones independientes. El método 
basado en \textit{clustering} espectro-temporal mostró mayor 
sensibilidad en escenarios de baja actividad, superando al 
enfoque de energías relativas. La aplicación de ambos algoritmos 
permitió detectar y asignar cada canto a su emisor. Del mismo 
modo, estas herramientas quedan disponibles para futuros 
estudios.

Se infirieron dos modelos de interacciones acústicas: el modelo 
independiente y el modelo de Ising. El segundo de ellos reveló 
acoplamientos significativos entre individuos, 
evidenciando una estructura de red con tendencias tanto a la 
co-emisión como a la inhibición. Sin embargo, la comparación con 
el modelo independiente sugirió que las interacciones podrían 
ser débiles.  Esto puede ocurrir porque los patrones se manifiestan 
con poca frecuencia 
en los datos, lo que los hace difíciles de predecir, 
o porque el modelo de Ising estándar 
no captura plenamente la dinámica no lineal del sistema. Lo anterior 
subraya la necesidad de incorporar términos de orden superior o 
dinámicas temporales explícitas en futuras aproximaciones. 
La validación manual estuvo limitada por la ausencia de un 
\textit{ground truth} experto, lo que resalta la importancia la 
importancia de mejorar 
los protocolos de etiquetado para reducir sesgos en la 
evaluación de algoritmos.

El presente trabajo representa un avance significativo en el 
estudio de \emph{Eleutherodactylus eileenae} en Cuba, al 
introducir un flujo completamente automatizado para el 
procesamiento de las grabaciones de campo,
capaz de identificar distintos individuos dentro de una misma 
especie de anuros, en contraste con la mayoría de estudios 
previos que se han centrado en distinguir especies distintas a 
partir de sus vocalizaciones.
Además, se propuso un enfoque novedoso para modelar 
matemáticamente las interacciones acústicas del coro mediante un 
sistema de espines. Con ello se facilita el análisis 
reproducible de grandes volúmenes de datos bioacústicos y se 
sienta una base cuantitativa para investigar la dinámica 
colectiva de los machos cantores.