\chapter{Antecedentes}\label{chapter: antecedentes}


Las primeras grabaciones sistemáticas de sonidos de vida 
silvestre se realizaron en mayo de 1929, cuando Arthur A. 
Allen y Peter P. Kellogg registraron por primera vez el canto 
de especies neárticas en un parque de Ithaca, Nueva York 
\cite{macaulaylibrary1929}. Este hito marcó el nacimiento de 
la colección que hoy conocemos como Macaulay Library, el 
mayor archivo multimedia de fauna del mundo, con más de 2.6 
millones de grabaciones de audio y cobertura de casi todas 
las especies de aves, así como registros de insectos, peces, 
anfibios y mamíferos \cite{allaboutbirds1929}.

Durante décadas, la bioacústica de campo dependió de hardware 
analógico y procesamiento manual de cintas, lo cual limitaba 
el volumen de datos y la reproducibilidad de los estudios 
\cite{blumstein2011acoustic}. Con la transición a sistemas digitales 
a principios de los años 2000, la monitorización acústica 
pasiva (PAM) emergió como herramienta clave en ecología y 
conservación, permitiendo grabaciones continuas y remotas que 
facilitan el muestreo a gran escala sin perturbar el 
comportamiento natural de los organismos \cite{acevedo2009automated}.

Las organizaciones de conservación, entre ellas WWF, 
impulsaron la adopción de PAM en áreas protegidas, 
destacando la necesidad de protocolos robustos de 
preprocesamiento para atenuar ruido ambiental (como lluvia, 
viento y tráfico) y mejorar la relación señal-ruido 
\cite{blumstein2011acoustic}. Paralelamente, el desarrollo de 
índices acústicos automatizados, introducidos por Towsey et 
al. (2014), permitió estimar la riqueza de especies en 
paisajes sonoros complejos mediante métricas computables 
directamente de las grabaciones \cite{towsey2014use}.

La creación de grandes conjuntos de datos de referencia, 
como AnuraSet para llamadas de anuros neotropicales, 
estandarizó la evaluación de algoritmos de identificación en 
entornos ruidosos, facilitando la comparación de metodologías 
y la reproducibilidad de los resultados 
\cite{canas2023dataset}. Revisiones como 
la de Stowell (2022) han descrito las oportunidades y 
desafíos de incorporar redes neuronales profundas en 
bioacústica, estableciendo una hoja de ruta para futuros 
avances computacionales \cite{stowell2022computational}.

En el ámbito del modelado de interacciones, Ōta et al. (2020) 
cuantificaron los mecanismos de sincronización en coros de 
ranas a partir de características dinámicas de las llamadas, 
demostrando la aplicabilidad de enfoques de red para estudiar 
la co-emisión y la causalidad acústica 
\cite{ota2020interaction}. Asimismo, Kalra (2024) examinó la 
percepción de señales en ambientes complejos, proporcionando 
métodos de localización pasiva que mejoran la atribución de 
fuentes en PAM \cite{kalra2024signal}.

Estos avances tecnológicos y metodológicos forman el 
fundamento de la presente investigación, que adapta flujos 
computacionales y modelos estadísticos de interacción al 
estudio de los coros de \emph{Eleutherodactylus eileenae} en 
su entorno natural cubano.



% region epígrafe 1
\section{Detección de individuos en grabaciones}
\label{sec:deteccion_individuos}



La detección precisa de emisores individuales en grabaciones 
de campo constituye un paso esencial para cualquier estudio 
bioacústico que aspire a caracterizar patrones de 
comportamiento a nivel de individuos. Este proceso implica la 
identificación de eventos acústicos relevantes en señales, 
la discriminación de llamadas frente a ruido ambiental y, 
cuando es posible, la atribución de cada emisión a su emisor 
de origen. A continuación, se revisan primero los métodos 
clásicos de clasificación de señales basados en análisis 
espectral y extracción de características, y luego se exponen 
enfoques de localización pasiva y heurísticas de asignación 
que permiten estimar la procedencia de las llamadas.

\subsection{Métodos clásicos de clasificación de señales acústicas}

La clasificación de vocalizaciones animales en grabaciones 
se fundamenta en la extracción de características 
espectrales a partir de la Transformada Rápida de Fourier 
(FFT) \cite{cooley1965algorithm}. Sobre esta base, la conversión a 
coeficientes cepstrales en frecuencia mel (MFCC) ha 
demostrado ser especialmente eficaz para representar texturas 
acústicas complejas, tal como lo estableció Mermelstein 
(1976) en el ámbito de voz humana y luego adaptado a 
bioacústica \cite{mermelstein1976distance}. Zhang et al. (2021) 
emplearon la descomposición de espectrogramas mel y la fusión de 
modelos para clasificar escenas acústicas, alcanzando mejoras 
significativas en la precisión de clasificación 
\cite{zhang2021acoustic}. Xie et al. (2016) extendieron este 
enfoque integrando detección de eventos acústicos con 
aprendizaje multi-etiqueta sobre espectrogramas, lo que 
permitió discriminar actividades de canto de ruido de lluvia 
con resultados prometedores \cite{xie2016detecting}. 
Estos métodos clásicos forman la columna vertebral de los 
flujos heurísticos de detección, siendo especialmente 
adecuados cuando solo se dispone de una única pista de audio.

\subsection{Localización y asignación de fuentes}
Aunque la verdadera localización tridimensional de emisores 
requiere múltiples canales sincronizados, diversos autores 
han explorado técnicas de triangulación y diferencias de 
tiempo de llegada (TDOA) para estimar direcciones de origen. 
Spiesberger y Fristrup (1990) detallaron un método de TDOA 
que calcula desplazamientos de fase entre pares de micrófonos 
para inferir la dirección de la fuente sonora 
\cite{spiesberger1990passive}. Cuando solo se cuenta 
con grabaciones monofónicas, se han diseñado heurísticas 
basadas en la intensidad relativa de la señal y la similitud 
espectral. Kalra (2024) demostró que, al comparar la energía 
de una llamada en momentos consecutivos y su correlación 
espectral con plantillas predefinidas, es posible asignar 
eventos a individuos cercanos con una precisión alta
\cite{kalra2024signal}. Estas heurísticas aprovechan la atenuación 
del sonido en el espacio y la coherencia espectral para 
discriminar entre vocalizaciones de distintos machos dentro 
de una misma grabación, constituyendo una alternativa 
pragmática cuando no se dispone de redes de micrófonos 
múltiples.



% region epígrafe 2
\section{Análisis de coros y comportamiento grupal}
\label{sec:analisis_coros}

El estudio de los coros de anuros combina observaciones de 
campo con análisis cuantitativos para desentrañar cómo las 
interacciones individuales dan lugar a patrones colectivos de 
canto. Este enfoque implica medir la distribución espacial y 
temporal de las vocalizaciones, identificar picos de actividad 
sincronizada y describir las reglas de proximidad y fase que 
los machos siguen al unirse al coro. A continuación se revisan 
los principales hallazgos sobre la formación de coros y las 
herramientas ecoacústicas desarrolladas para caracterizar su 
estructura y dinámica.


\subsection{Patrones de formación de coros} 
En \emph{Eleutherodactylus coqui}, Woolbright (1985) registró 
que los individuos ascienden al anochecer a perchas de la 
vegetación, donde emiten series regulares de “co-qui” 
alternadas con períodos de silencio, y observó que la 
densidad de cantores influye en la cercanía entre perchas, 
reduciéndose la distancia media en noches de alta actividad 
\cite{woolbright1985patterns}. Stewart y Pough (1983) encontraron en 
la misma especie que la localización de los machos en el 
espacio de canto está relacionada con la disponibilidad de 
refugios y la competencia territorial, ajustando su posición 
en respuesta a la presencia de vecinos \cite{stewart1983population}. 
Drewry y Rand (1983) describieron patrones similares en 
\emph{E. coqui} en Puerto Rico, donde la sincronización de 
picos vocales ocurre en respuesta a variaciones de luz y 
temperatura, con crestas al anochecer y al amanecer 
\cite{drewry1983characteristics}.  

En el caso de \emph{Eleutherodactylus eileenae} en Cuba, 
Alonso et al. (2001) analizaron la actividad acústica de 
machos cantores y revelaron dos picos de emisión al anochecer 
y al amanecer, así como desplazamientos verticales en la 
vegetación que enfatizan la dimensión espacial de la formación 
de coros \cite{alonso2001patrones}. Estos hallazgos concuerdan con 
los patrones descritos en otras especies de \emph{Eleutherodactylus}, 
donde la competencia acústica y las condiciones 
microambientales determinan la estructura del coro 
\cite{townsend1994reproductive}. Más recientemente, Ōta et al. (2020) 
cuantificaron dinámicamente las interacciones entre pares de 
individuos mediante análisis de series temporales de llamadas, 
demostrando que algunos machos actúan como “iniciadores” de 
pulsos colectivos, un comportamiento que puede describirse 
con modelos de red de interacción \cite{ota2020interaction}.  

\subsection{Ecoacústica de coros}
La ecoacústica de coros se apoya en el uso de índices 
acústicos y herramientas computacionales para detectar y 
caracterizar la actividad colectiva sin necesidad de 
seguimiento visual. Boelman et al. (2007) introdujeron el
Bioacoustic Index (BI), calculado como la energía acumulada 
en la banda de frecuencias de interés \cite{boelman2007multi}, 
mientras que Pieretti et al. (2011) complementaron este enfoque con el 
Acoustic Complexity Index (ACI), que mide variaciones 
espectrales dentro de ventanas temporales para estimar la 
riqueza y la intensidad coral, mostrando alta correlación con 
el número de machos activos en grabaciones de bosques 
tropicales, y demostraron que ambos 
índices permiten resaltar 
los períodos de mayor sincronización en coros de anfibios 
\cite{pieretti2011new}. 
Además, Farina (2018) estableció un marco cuantitativo para la 
ecoacústica que enfatiza la integración de índices acústicos en 
estudios de dinámica comunitaria, subrayando su utilidad para 
evaluar patrones de sincronización en coros de anuros y otros 
sistemas biológicos \cite{farina2018ecoacoustics}.

En 2021, Gan et al. propusieron un método de reconocimiento 
de coros basado en espectrogramas falsos en color y 
aprendizaje automático, logrando identificar patrones de 
emisión colectiva con una precisión del 77 \% en grabaciones 
de campo de especies tropicales \cite{gan2021novel}. Brodie et al. 
(2020) evaluaron la eficacia de múltiples índices acústicos y 
confirmaron que la combinación de ACI, BI y Net Spectrum 
ofrece una representación robusta de la estructura coral a lo 
largo de ciclos nocturnos \cite{brodie2020automated}.


% region epígrafe 3
\section{Modelos de interacción y causalidad}
\label{sec:modelos_interaccion}

En ciencias biológicas y ecológicas, comprender cómo las 
interacciones locales entre individuos dan lugar a fenómenos 
colectivos constituye un desafío central. Los modelos 
estadísticos de redes, en particular las formulaciones de 
máxima entropía equivalentes al modelo de Ising, ofrecen un 
marco potente para describir la sincronización y la causalidad 
en sistemas de canto animal. En esta sección se revisan primero 
las bases y aplicaciones de los modelos de Ising en redes 
biológicas, y luego se presentan las principales técnicas de 
inferencia de causalidad adaptadas a series temporales acústicas.

\subsection{Modelos estadísticos de redes}
El modelo de Ising, originado en física estadística para 
describir ferromagnetismo, se reformuló como un problema de 
máxima entropía para ajustar distribuciones observadas de 
estados binarios con correlaciones dadas. Schneidman et al. 
(2006) demostraron que un modelo de Ising con interacciones 
par a par reproduce con alta fidelidad las correlaciones de 
actividad en poblaciones de neuronas retinianas, sin necesidad 
de términos de orden superior \cite{schneidman2006weak}. Mora y 
Bialek (2011) ampliaron esta perspectiva, argumentando que 
sistemas biológicos (desde redes neuronales hasta colonias de 
insectos) operan cerca de puntos críticos de fase, un rasgo que 
maximiza la capacidad de información y la sensibilidad a 
perturbaciones externas \cite{mora2011biological}.  

Para inferir los parámetros de interacción \(J_{ij}\) en 
sistemas fuera de equilibrio, Zeng et al. (2013) desarrollaron 
un método de máxima verosimilitud para modelos de Ising 
asíncronos, derivando reglas de aprendizaje que dependen 
únicamente de las correlaciones temporales de los datos 
\cite{zeng2013maximum}. Chau Nguyen et al. (2017) revisaron un 
amplio espectro de métodos de inferencia inversa (incluyendo 
pseudoverosimilitud y aproximaciones de mensaje) que escalan 
eficientemente a grandes redes biológicas, destacando 
aplicaciones en neurociencia y ecología \cite{chau2017inverse}. 
Estos enfoques han permitido reconstruir estructuras de 
interacción tanto en poblaciones de neuronas 
(Tkačik et al., 2006) como en sistemas de canto colectivo, donde 
las “conexiones” cuantifican la probabilidad de co-emisión entre 
pares de individuos \cite{tkacik2006ising}.

\subsection{Técnicas de inferencia de causalidad}
Más allá de la correlación, establecer relaciones causales en 
series temporales acústicas requiere herramientas específicas. 
El enfoque clásico de Granger (1969) define la causalidad en 
términos predictivos: una serie \(X\) "cause" a otra \(Y\) si 
el pasado de \(X\) mejora la predicción de \(Y\) más allá de lo 
que hace el propio pasado de \(Y\) \cite{granger1969investigating}. 
Porta y Faes (2015) ampliaron este marco en el contexto de 
sistemas fisiológicos y redes neuronales, describiendo cómo la 
causalidad de Wiener-Granger puede aplicarse a señales 
biológicas con metodologías que consideran tanto ruido como 
dinámicas no lineales \cite{porta2015wiener}.  


Para capturar relaciones más complejas, se han empleado medidas de 
información direccional como la entropía de transferencia 
(Schreiber, 2000), que cuantifica flujos de información sin 
asumir linealidad \cite{schreiber2000measuring}. Más recientemente, el 
método de Convergent Cross Mapping (Sugihara et al., 2012) ha 
mostrado robustez al inferir causalidad en sistemas dinámicos de 
alta dimensionalidad, al reconstruir espacios de fase conjuntos 
y medir la habilidad de uno de los vectores de recuperación del 
otro \cite{sugihara2012detecting}. Estas técnicas, que serán
utilizadas en trabajos futuros, proporcionan un 
conjunto de herramientas complementarias para analizar la 
causalidad acústica en coros de \emph{E.\,eileenae}, permitiendo 
distinguir interacciones directas de sincronizaciones inducidas 
por factores ambientales comunes.

% region epígrafe 4
\section{Antecedentes bioacústicos sobre \emph{E.\,eileenae}}
\label{sec:antecedentes_eileenae}

La investigación acerca de la biología y ecología de 
\emph{Eleutherodactylus eileenae} arrancó fundamentalmente con 
estudios taxonómicos y de distribución. Dunn (1926) 
describió por primera vez la especie a partir de individuos 
recolectados en localidades de la provincia de Cienfuegos
fundamentalmente (Soledad, Mina Carlota, Guane y Hoyo Colorado) \cite{dunn1926additional}. 
Posteriormente, 
Estrada (1984, 1994) amplió el conocimiento de su rango 
geográfico, documentando nuevas localidades en Sancti Spíritus 
y otros municipios centrales, así como preferencias de hábitat 
que incluyen bosques semideciduos, pinares y cafetales hasta 
los 900 m s.\,n.\,m. \cite{estrada1984nuevas,estrada1994herpetofauna}. 
Estas obras sentaron las bases para comprender la distribución 
espacial de \emph{E.\,eileenae} y su adaptación a distintos 
sustratos, como bromelias y hojarasca.

En el plano de la bioacústica y la ecología trófica, Alonso et 
al. (2001) llevaron a cabo muestreos en la Reserva de la 
Biosfera “Sierra del Rosario”, donde cuantificaron patrones estacionales 
de vocalización de los machos y 
relacionaron la actividad acústica con la alimentación mediante 
análisis de contenido estomacal \cite{alonso2001patrones}. Encontraron 
dos picos de canto, al anochecer y al amanecer, coincidentes 
con desplazamientos verticales en la vegetación, y demostraron 
que durante los períodos de depresión vocal los machos se 
alimentan, consumiendo principalmente himenópteros y arácnidos. 
Este estudio fue el primero en integrar datos acústicos y 
tróficos de \emph{E.\,eileenae}, estableciendo el marco 
experimental y metodológico que la presente tesis extiende 
mediante flujos computacionales automatizados y modelado de 
interacciones acústicas. Díaz et al. (2006) y más tarde Díaz y Cádiz 
(2008) ofrecieron descripciones más detalladas de las llamadas 
de anuncio de los machos de la especie, pero coincidentes con 
las de Alonso Bosch y Rodríguez (2001) \cite{diaz2006guia,diaz2008guia,alonso2001llamadas}.
















