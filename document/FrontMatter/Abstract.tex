\begin{abstract}
    El análisis computacional de coros animales representa un 
desafío en ecoacústica, particularmente en ambientes con 
superposición de señales y ruido. Esta investigación 
desarrolló un flujo automatizado para procesar grabaciones 
de \emph{Eleutherodactylus eileenae}, una rana endémica de 
Cuba cuyos machos adultos forman coros nocturnos para atraer a las hembras para la reproducción. 
Se aplicó un filtrado espectral basado en percentiles 
para atenuar interferencias de banda ancha y puntuales. A continuación, 
nueve pistas de audio se sincronizaron mediante correlación cruzada. 
Se implementaron dos algoritmos heurísticos para detectar y asignar 
cantos a micrófonos: uno basado en hipótesis de energías relativas a los audios 
y otro mediante \emph{clustering} espectro-temporal, este último 
mostrando mayor precisión en condiciones de baja actividad. Las series 
temporales resultantes se modelaron como una red de espines usando 
el modelo de Ising, cuyos parámetros \(J_{ij}\) se inferieron por 
máxima verosimilitud. Aunque se identificaron acoplamientos significativos, 
el modelo independiente superó al de Ising en la predicción de patrones, 
sugiriendo interacciones débiles o limitaciones en la formulación estándar 
del modelo. El trabajo demuestra que enfoques heurísticos reproducibles 
permiten reconstruir dinámicas colectivas en coros y propone mejoras 
metodológicas para futuros estudios.
\end{abstract}

\begin{enabstract}
The computational analysis of animal choruses poses a significant 
challenge in ecoacoustics, particularly in environments with 
signal overlap and noise. In this study, we built an automated 
pipeline to process recordings of \emph{Eleutherodactylus 
eileenae}, a Cuban endemic frog whose adult males form nocturnal 
choruses to attract females for reproduction. We first applied percentile-based 
spectral filtering to attenuate broadband and transient 
interference, and then synchronized the nine audio tracks with 
cross-correlation. We designed two heuristic algorithms to 
detect and assign calls to microphones: one relied on 
relative-energy hypotheses, while the other used spectro-temporal 
clustering; the latter achieved higher accuracy under low vocal 
activity. We modeled the resulting time series as a spin network 
with the Ising framework and inferred the interaction parameters 
\(J_{ij}\) through maximum likelihood. Although the Ising model 
revealed significant couplings, an independent model better 
predicted call patterns, indicating either weak interactions or 
limitations in the standard formulation. Overall, our 
reproducible heuristic workflow reconstructs collective 
dynamics in animal choruses and points to methodological 
refinements for future research.
\end{enabstract}