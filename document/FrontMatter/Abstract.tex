\begin{abstract}
    El análisis computacional de coros animales representa un 
desafío en ecoacústica, particularmente en ambientes con 
superposición de señales y ruido. Esta investigación 
desarrolló un flujo automatizado para procesar grabaciones 
de \emph{Eleutherodactylus eileenae}, una rana endémica de 
Cuba cuyos machos forman coros nocturnos para atraer a las hembras. 
Se aplicó un filtrado espectral basado en percentiles 
para atenuar interferencias de banda ancha y puntuales. A continuación, 
nueve pistas de audio se sincronizaron mediante correlación cruzada. 
Se implementaron dos algoritmos heurísticos para detectar y asignar 
cantos a micrófonos: uno basado en hipótesis de energías relativas a los audios 
y otro mediante \emph{clustering} espectro-temporal, este último 
mostrando mayor precisión en condiciones de baja actividad. Las series 
temporales resultantes se modelaron como una red de espines usando 
el modelo de Ising, cuyos parámetros \(J_{ij}\) se inferieron por 
máxima verosimilitud. Aunque se identificaron acoplamientos significativos, 
el modelo independiente superó al de Ising en la predicción de patrones, 
sugiriendo interacciones débiles o limitaciones en la formulación estándar 
del modelo. El trabajo demuestra que enfoques heurísticos reproducibles 
permiten reconstruir dinámicas colectivas en coros y propone mejoras 
metodológicas para futuros estudios.
\end{abstract}

\begin{enabstract}
The computational analysis of animal choruses poses a significant 
challenge in ecoacoustics, particularly in environments with 
signal overlap and noise. This study developed an automated pipeline 
to process recordings of \emph{Eleutherodactylus eileenae}, a frog 
endemic to Cuba whose males form nocturnal choruses to attract females. 
First, spectral filtering based on amplitude percentiles was applied 
to attenuate both broadband and transient interference. Then, nine audio 
tracks were synchronized via cross-correlation. Two heuristic algorithms 
were implemented to detect and assign calls to microphones: one based 
on hypotheses of relative energy in the recordings and another using 
spectro-temporal clustering, the latter demonstrating higher accuracy 
under low activity conditions. The resulting time series were modeled 
as a spin network using the Ising model, with interaction parameters 
\(J_{ij}\) inferred via maximum likelihood. Although significant 
couplings were identified, an independent model outperformed the 
Ising model in predicting call patterns, suggesting either weak 
interactions or limitations in the standard model formulation. This 
work demonstrates that reproducible heuristic approaches can reconstruct 
collective dynamics in animal choruses and proposes methodological 
improvements for future studies.
\end{enabstract}